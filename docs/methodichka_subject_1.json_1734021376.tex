
\documentclass[a4paper,12pt]{article}
\usepackage[utf8]{inputenc}
\usepackage[english,russian]{babel}
\usepackage{graphicx}
\usepackage{amsmath}
\usepackage[T2A]{fontenc}
\textwidth=17cm
\oddsidemargin=0pt
\topmargin=-2cm
\textheight=27cm

\begin{document}
Дата: 2024-12-12

Подпись: \underline{\hspace{3cm}}

\section*{Методические рекомендации}
\begin{itemize}
\item в первом разделе уделить особое внимание различным видам записи законов сохранения и структуре решения газодинамической системы уравнений;
\item во втором разделе рассмотреть основные принципы исследования свойст разностных схем;
\item в третьем разделе особое внимание уделить консервативности разностных схем;
\item в четвертом разделе рассмотреть методику исследования устойчивости разностных схем для газодинамической системы уравнений;
\item в пятом разделе показать особенности реализации разностных схем для уравнений газодинамики;
\item в шестом -- остановиться на основных принципах выбора счетной сетки и построения, разностных схем для многомерных задач газовой динамики;
\item в седьмом рассмотреть разностные уравнении лежащие в основе старейшей лагранжевой методики Д;
\item в восьмом обратить внимание на достоинства и недостатки явных и неявных разностных схем;
\item в девятом рассмотреть основные принципы метода частиц;
\item в десятом уделить особое внимание методу концентраций.
\end{itemize}
\end{document}